\chapter{Exercices}
\begin{exercise}
    L'orbite quasi circulaire de la Lune autour de la Terre a un rayon d'environ 385 000[km] et une période de 27,3 jours.
    Détermine l'accélération centripète de la Lune causée par l'attraction de la Terre.
\end{exercise}

\begin{exercise}
    Une pièce de monnaie de \(3[g]\) est placée à \(17[cm]\) de l'axe d'un disque en rotation.
    La vitesse du disque est progressivement augmentée et la pièce commence à glisser lorsque la vitesse de rotation atteint 47 tours par minute.
    Quel est le coefficient de frottement entre la pièce et le disque ?
\end{exercise}
\begin{solution}
    \(\mu_s=0,4198\)
\end{solution}

\begin{exercise}
    Un satellite artificiel décrit un cercle autour de la Terre à \(270[km]\) du sol. À cette altitude, l'accélération gravitationnelle vaut \(9.7299[m \cdot s^{-2}]\).
    \begin{enumerate}[a]
        \item Détermine la vitesse de ce satellite.
        \item Détermine la période de ce satellite.
    \end{enumerate}
    Le rayon de la Terre, la distance entre le centre et la surface, est de 6400 [km].
\end{exercise}

\begin{exercise}
    Une corde de \(0,7[m]\) a une tension de rupture de \(38[N]\). Cela signifie qu'elle se rompt lorsque la force de tension qui s'exerce sur elle est supérieur à la tension de rupture. Une masse de \(400[g]\) est attachée à l'extrémité de cette corde. Quelle est la vitesse maximale avec laquelle on peut faire tourner la masse à l'aide de la corde ?
\end{exercise}
\begin{solution}
    \(v_max=8,154[m/s]\)
\end{solution}

\begin{exercise}
    Détermine la vitesse maximale avec laquelle une voiture de \(1000[kg]\) peut prendre un virage d'un rayon de \(80[m]\) sur une route plate si le coefficient de frottement entre les pneus et la route est de \(\mu _s=0,5\).
\end{exercise}
\begin{solution}
    \(v=19,809[m/s]\)
\end{solution}

\begin{exercise}
    À quel endroit d'un disque en rotation faut-il se placer pour avoir la plus grande vitesse angulaire ?
\end{exercise}
\begin{solution}
    La vitesse angulaire est la même partout sur le disque.
\end{solution}

\begin{exercise}
    À quel endroit d'un disque en rotation faut-il se placer pour avoir la plus grande vitesse tangentielle ?
\end{exercise}
\begin{solution}
    La vitesse tangentielle est la plus élevée près du bord du disque.
\end{solution}

%dans mon dossier exe_sup_MCU_tir_exam, j'ai plusieurs exercices que je pourrais ajouter ici