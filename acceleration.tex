\chapter{Accélération}
Dans un MCU, la valeur de la vitesse est constante, mais sa direction varie. Cette vitesse n'est donc pas constante et cela implique qu'il existe une accélération !
En physique, l'accélération est toujours définie comme la différence de vitesse entre deux instants, divisée par le temps qui sépare ces deux instants :
\begin{equation*}
    \vec{a}=\frac{\vec{v_1}-\vec{v_0}}{\Delta t}
\end{equation*}

L'accélération du MCU n'échappe pas à cette règle.
Toutefois, pour que l'accélération ainsi calculée soit une accélération instantanée et non une accélération moyenne, il faut que l'intervalle de temps entre \(t_1\) et \(t_0\) soit le plus proche possible de 0 (zéro).

\section{Sens et direction du vecteur accélération}
L'animation sur le calcul de l'accélération présentée en classe a mis en évidence que lorsque l'intervalle de temps est court, le vecteur accélération est dirigé radialement et vers le centre du cercle.

\begin{encadre}
    Dans un MCU, l'accélération est :
    \begin{itemize}[label=\textbullet]
        \item radiale : de même direction que les rayons du cercle ;
        \item \motcle{centripète} : elle est toujours dirigée vers le centre du cercle.
    \end{itemize}
\end{encadre}

\definecolor{ffqqqq}{rgb}{1,0,0}
\definecolor{qqzzqq}{rgb}{0,0.6,0}
\definecolor{ffxfqq}{rgb}{1,0.4980392156862745,0}
\definecolor{qqqqff}{rgb}{0,0,1}
\definecolor{uuuuuu}{rgb}{0.26666666666666666,0.26666666666666666,0.26666666666666666}
\begin{tikzpicture}[line cap=round,line join=round,>=triangle 45,x=1cm,y=1cm]
    \clip(-6,-6) rectangle (6,6);
    \draw [line width=2pt,color=qqqqff] (0,0) circle (5cm);
    \draw [line width=2pt,color=ffxfqq] (0,0)-- (4.235128716917126,2.6577593478612576);
    \draw [line width=2pt,color=ffxfqq] (0,0)-- (0.8944271909999164,4.919349550499537);
    \draw [->,line width=2pt,color=qqzzqq] (0.8944271909999164,4.919349550499537) -- (-3.0410524493997135,5.6348913032994705);
    \draw [->,line width=2pt,color=qqzzqq] (4.235128716917126,2.6577593478612576) -- (2.1089212386281204,6.045862321394958);
    \draw [->,line width=2pt,color=ffqqqq] (2.5647779539585214,3.7885544491803973) -- (0.7555057918478971,1.11599322844663);
    \begin{scriptsize}
        \draw [fill=uuuuuu] (0,0) circle (2pt);
        \draw[color=ffxfqq] (2.88,1.27) node {$r$};
        \draw[color=ffxfqq] (0.92,2.55) node {$r$};
        \draw[color=qqzzqq] (-0.32,5.75) node {\(\vec{v_{t2}}\)};
        \draw[color=qqzzqq] (4.04,4.71) node {\(\vec{v_{t1}}\)};
        \draw[color=ffqqqq] (2.82,2.75) node {\(\vec{a_c}\)};
    \end{scriptsize}
\end{tikzpicture}

\newpage

\section{Intensité de l'accélération}
Dans un MCU, l'accélération centripète vaut :
\begin{equation}
    a=\frac{\Delta v}{\Delta t}
\end{equation}

\definecolor{wwffqq}{rgb}{0.4,1,0}
\definecolor{zzttqq}{rgb}{0.6,0.2,0}
\definecolor{ttffqq}{rgb}{0.2,1,0}
\definecolor{ffxfqq}{rgb}{1,0.4980392156862745,0}
\definecolor{ffqqqq}{rgb}{1,0,0}
\definecolor{qqzzqq}{rgb}{0,0.6,0}
\definecolor{qqqqff}{rgb}{0,0,1}
\begin{tikzpicture}[line cap=round,line join=round,>=triangle 45,x=1cm,y=1cm]
    %\clip(-7.7,-5.4) rectangle (14.94,13.96);
    \fill[line width=0pt,color=zzttqq,fill=zzttqq,fill opacity=0.10000000149011612] (4.629422378491702,7.950168030534125) -- (3.1449065223887356,10.557122704666163) -- (1.6667963555911083,8.422234814402902) -- cycle;
    \fill[line width=0pt,color=wwffqq,fill=wwffqq,fill opacity=0.1] (4,4) -- (4.629422378491702,7.950168030534125) -- (7.475939565509384,5.979354474803954) -- cycle;
    \draw [line width=2pt,color=qqqqff] (4,4) circle (4cm);
    \draw [line width=2pt,color=qqzzqq] (4,4)-- (7.475939565509384,5.979354474803954);
    \draw [line width=2pt,color=qqzzqq] (4,4)-- (4.629422378491702,7.950168030534125);
    \draw [->,line width=2pt,color=ffqqqq] (7.475939565509384,5.979354474803954) -- (5.991423709406417,8.586309148935992);
    \draw [->,line width=2pt,color=ffqqqq] (4.629422378491702,7.950168030534125) -- (1.6667963555911083,8.422234814402902);
    \draw [->,line width=2pt,dash pattern=on 1pt off 1pt,color=ffqqqq] (4.629422378491702,7.950168030534125) -- (3.1449065223887356,10.557122704666163);
    \draw [->,line width=2pt,color=ffxfqq] (3.1449065223887356,10.557122704666163) -- (1.6667963555911083,8.422234814402902);
    \draw [line width=2pt,color=ttffqq] (7.475939565509384,5.979354474803954)-- (4.629422378491702,7.950168030534125);
    \begin{scriptsize}
        \draw [fill=black] (4,4) circle (2.5pt);
        \draw[color=black] (4.38,3.91) node {C};
        \draw [fill=black] (7.475939565509384,5.979354474803954) circle (2.5pt);
        \draw[color=black] (7.84,6.17) node {\(x_1\)};
        \draw [fill=black] (4.629422378491702,7.950168030534125) circle (2.5pt);
        \draw[color=black] (5.02,8.25) node {\(x_2\)};
        \draw[color=qqzzqq] (5.96,4.95) node {r};
        \draw[color=qqzzqq] (4.7,6.17) node {r};
        \draw[color=ffqqqq] (7.34,7.55) node {\(\vec{v_1}\)};
        \draw[color=ffqqqq] (2.76,8.5) node {\(\vec{v_2}\)};
        \draw[color=ffqqqq] (4,9.51) node {\(\vec{v_1}\)};
        \draw[color=ffxfqq] (1.5,10) node {\(\vec{\Delta v} = \vec{v_2} - \vec{v_1}\)};
    \end{scriptsize}
\end{tikzpicture}

Or, le triangle formé par les vecteurs \(\vec{v_1} ~ ; ~ \vec{v_2} ~ et ~ \vec{\Delta v}\) (en rouge) et le triangle formé par les deux rayons \(r\) (en vert) sont des triangles semblables.
On a donc le droit d'écrire que :
\begin{equation}
    \frac{\Delta v}{b} = \frac{v}{r}
    \label{eqn:triangles_semblables}
\end{equation}
Où \(b\) est le segment reliant les points \(x_1\) et \(x_2\).

\newpage

Lorsque l'intervalle de temps, \(\Delta t\), qui sépare les points \(x_1\) et \(x_2\) devient très petit, l'arc de cercle entre \(x_1\) et \(x_2\) (\(l\)) à la même valeur que le segment de droite entre \(x_1\) et \(x_2\), ()\(b\)).

Or, la valeur, \(l\),de l'arc de cercle est connue, elle vaut :
\begin{equation}
    l=v \cdot \Delta t
    \label{eqn:longueur_arc}
\end{equation}

En associant les équations \ref{eqn:triangles_semblables} et \ref{eqn:longueur_arc}, on a alors le droit d'écrire que :
\begin{equation}
    \frac{\Delta v}{v \cdot \Delta t} = \frac{v}{r}
\end{equation}

On peur réarranger cette équation pour obtenir :
\begin{equation}
    \frac{\Delta v}{\Delta t}=\frac{v^2}{r}
\end{equation}

Et comme \(a_c=\frac{\Delta v}{\Delta t}\) :
\begin{equation}
    a_c=\frac{v^2}{r}
\end{equation}

\begin{mytheo*}{Accélération centripète}
    \begin{equation}
        a_c=\frac{v^2}{r}
    \end{equation}
\end{mytheo*}

\begin{itemize}[label=\(\rightarrow\)]
    \item Quelle relation mathématique permet de calculer l'accélération centripète à partir de la vitesse angulaire ?
          \pointilles{2}
\end{itemize}

\newpage

\section{Exercices}
\begin{exercise}
    Une personne fait tourner une fronde avec une fréquence de \(5[Hz]\). La longueur du fil est de \(70[cm]\) et la masse en mouvement est de \(100[g]\).
    \begin{enumerate}[label=\alph*]
        \item Quelle est la valeur de l'accélération centripète.
        \item Quelle est l'intensité de la force avec laquelle il faut tirer sur le fil.
        \item Lorsque l'objet en mouvement est libéré de la fronde, dans quelle direction part-il ? Fait un schéma \enquote{vu de haut}.
    \end{enumerate}
\end{exercise}
\begin{solution}
    \begin{enumerate}[label=\alph*]
        \item \(a_c=70 \pi^2[m \cdot s^{-2}]\)
        \item \(F_c=7 \cdot \pi^2[N] \)
        \item Il se déplace en suivant un MRU
    \end{enumerate}
\end{solution}

\begin{exercise}
    Une force centripète constante de 3[N] s'exerce sur un objet de 5[kg] se déplaçant à la vitesse de 12[m/s].
    \begin{enumerate}[label=\alph*]
        \item Comment l'objet va-t-il se comporter ?
        \item Quel sera le rayon de sa trajectoire ?
        \item Combien de temps faudra-t-il pour faire un tour complet ?
    \end{enumerate}
\end{exercise}
\begin{solution}
    \begin{enumerate}[label=\alph*]
        \item Il va commencer à tourner et son mouvement sera celui d'un MCU.
        \item \(r=240[m] \)
        \item \(T=40 \pi[s]\)
    \end{enumerate}
\end{solution}

\begin{exercise}
    Une voiture de \(300[kg]\) roule à la vitesse de \(14,4[km/h]\) autour d'un rond-point dont le rayon vaut \(5[m]\).
    \begin{enumerate}[a)]
        \item Quelle est la nature de la force qui induit l'accélération centripète ?
        \item Que vaut l'accélération centripète ?
    \end{enumerate}
\end{exercise}
\begin{solution}
    \begin{enumerate}[label=\alph*]
        \item Il s'agit des forces de frottement entre les pneus et la route .
        \item \(a_c=3,2[m \cdot s^{-2}] \)
    \end{enumerate}
\end{solution}

\begin{exercise}
    Avec quelle force faut-il tirer sur un objet de 10[kg] se déplaçant à la vitesse de 54[km/h] pour qu'il fasse un tour complet en 30[s]. Quelle sera la longueur de la trajectoire ?
\end{exercise}
\begin{solution}
    \begin{enumerate}[label=\alph*]
        \item \(F_c=10 \pi[N]\)
        \item \(l=450[m] \)
    \end{enumerate}
\end{solution}

\newpage

\begin{exercise}
    Quelle est l'accélération centripète d'un mobile décrivant un cercle de rayon \(r=60[cm]\) à vitesse constante et à une fréquence de \(f=0.86826[Hz]\) ?
\end{exercise}
\begin{solution}
    \(a_c=17,857[m/s^2]\)
\end{solution}
\begin{exercise}
    Quelle est la vitesse angulaire d'un mobile se déplaçant selon un MCU dont le rayon vaut \(r=70[cm]\) et l'accélération centripète \(a=12.5[m \cdot s^{-2}]\)?
\end{exercise}
\begin{solution}
    \(\omega=4,2258 [rad/s]\)
\end{solution}
\begin{exercise}
    Quelle est la vitesse tangentielle d'un mobile en MCU s'il prend \(1.4735[s]\) pour accomplir un tour dont le rayon vaut \(r=110[cm]\)?
\end{exercise}
\begin{solution}
    \(v=4,6904 [m/s]\)
\end{solution}
\begin{exercise}
    À quelle fréquence tourne un mobile de \(420[g]\) attaché au bout d'une corde de \(70[cm]\) de long si la force qui tire sur le fil est de \(F=6[N]\)?
\end{exercise}
\begin{solution}
    \(f=0.71899[Hz]\)
\end{solution}

