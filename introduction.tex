\chapter{Introduction}
Jusqu'à présent, deux types de mouvements ont été abordés dans le cadre du cours de physique :
\begin{itemize}
      \item le MRU et le repos qui sont équivalents,
      \item le MRUA.
\end{itemize}

Il existe bien d'autres types de mouvements et ce chapitre sera consacré au \textbf{mouvement circulaire uniforme}.

\begin{encadre}
      Un corps décrit un mouvement circulaire uniforme si :
      \begin{itemize}
            \item sa trajectoire est circulaire,
            \item l'intensité de sa vitesse est constante.
      \end{itemize}
\end{encadre}

\begin{figure}[h!]
      \centering

      \begin{tikzpicture}[line cap=round,line join=round,>=triangle 45,x=1cm,y=1cm]
            \tkzDefPoint(0,0){O}
            \tkzDefPoint(2.5,0){A}
            \tkzDefPointBy[rotation=center O angle 50](A)
            \tkzGetPoint{B}

            \tkzFillAngle[size=1cm,left color=white,right color=YellowOrange](A,O,B)
            \tkzMarkAngle[size = 1cm,color=Orange,mark=none,line width=1pt](A,O,B)


            \tkzDrawSegment[line width=2pt,color=red](O,A)
            \tkzDrawSegment[line width=2pt,color=OliveGreen](O,B)
            \tkzDrawCircle[line width=2pt,color=blue](O,A)

            \tkzLabelPoint[right=2pt,color=OliveGreen](B){$B$}
            \tkzLabelSegment[auto,color=OliveGreen](O,B){$r$}
      \end{tikzpicture}
      \caption{Si le disque tourne à vitesse constante, le point \(B\) effectue un MCU}
\end{figure}

\newpage

\section{Période et fréquence}
Un exemple basique de MCU est donné par l'observation d'un point situé au bord d'un disque en rotation.
Le temps pris pour faire un tour complet est appelé \motcle{période}.
\begin{encadre}
      \motcle{Période} : \(T [s]\)
\end{encadre}



\begin{itemize}[label=\(\rightarrow\)]
      \item Un objet décrit un MCU dont la période vaut \(T=5[s]\). Combien de tour fait-il en \(1[s]\) ?
            \pointilles{2}
\end{itemize}
Le nombre de tours parcourus en \(1[s]\) pour un corps en rotation est appelé \motcle{fréquence}.
\begin{encadre}
      \motcle{Fréquence} : \(f [\frac{1}{s}] ; [s^{-1}] ; [Hz]\)
\end{encadre}

\begin{itemize}[label=\(\rightarrow\)]
      \item Quelle relation mathématique permet de calculer la période à partir de la fréquence ?
            \pointilles{2}
\end{itemize}

\newpage

\section{Vitesse tangentielle}
La vitesse moyenne entre deux points a été définie comme :
\(v_{moy}=\frac{\Delta x}{\Delta t}\)
On observe que lorsque l'intervalle de temps entre \(x_1\) et \(x_2\) est très court, le vecteur vitesse devient tangent à la trajectoire, c'est-à-dire perpendiculaire au rayon.


\begin{figure}[h!]
      \centering
            \begin{tikzpicture}[line cap=round,line join=round,>=triangle 45,x=1cm,y=1cm]
            \tkzDefPoint(0,0){O}
            \tkzDefPoint(2.5,0){A}
            \tkzDefPointBy[rotation=center O angle 50](A)
            \tkzGetPoint{B}

            \tkzFillAngle[size=1cm,left color=white,right color=YellowOrange](A,O,B)
            \tkzMarkAngle[size = 1cm,color=Orange,mark=none,line width=1pt](A,O,B)


            \tkzDrawSegment[line width=2pt,color=red](O,A)
            \tkzDrawSegment[line width=2pt,color=OliveGreen](O,B)
            \tkzDrawCircle[line width=2pt,color=blue](O,A)
            \tkzDefTangent[at=B](O)
            \tkzGetPoint{H}
            %\tkzDrawSegments[vect](B,H)
            \tkzDrawLine[->,line width=1.5pt,color=BrickRed,add=0 and 2](B,H)

            \tkzLabelPoint[right=2pt,color=OliveGreen](B){$B$}
            \tkzLabelSegment[auto,color=OliveGreen](O,B){$r$}
            \tkzMarkRightAngle[size=.3,fill=lightgray,opacity=.5](O,B,H)
            \tkzLabelSegment[above,color=BrickRed](B,H){$\vec{v}$}
      \end{tikzpicture}

      \caption{Le vecteur vitesse est tangent à la trajectoire.}
      \label{vitesse_tangentielle}
\end{figure}

\subsection{Intensité de la vitesse}
La circonférence d'un cercle étant égale à \(c=2 \cdot \pi \cdot r\), la valeur de la vitesse tangentielle dans un MCU est donnée par :
\begin{mytheo*}{Vitesse tangentielle}
      \begin{equation}
            v=\frac{2 \cdot \pi \cdot r}{T}
      \end{equation}
\end{mytheo*}


\newpage

\section{Exercices}
\begin{exercise}
      La distance entre la Terre et la Lune est de 384 000[km] et elle prend 27,32 jours pour faire un tour complet. Si on considère que le mouvement de la Lune est un MCU, calcule la vitesse de la Lune.
\end{exercise}
\begin{solution}
      \(v=1022[m/s]\)
\end{solution}

\begin{exercise}
      Une roue dont le rayon vaut 30 cm tourne avec une fréquence de 50[Hz]. Quelle est la vitesse d'un caillou situé sur le bord de celle-ci ?
      Si le caillou se décroche, par où va-t-il partir, fais un schéma.
\end{exercise}
\begin{solution}
      \(v=94,25[m/s]\)
\end{solution}

\begin{exercise}
      Une voiture s'engage sur un rond-point à la vitesse de 20[km/h]. Elle prend 3[s] pour franchir la première moité de celui-ci. Quelle est la longueur du rayon du rond-point ?
\end{exercise}
\begin{solution}
      \(r=5,3052[m]\)
\end{solution}

\newpage

\section{Vitesse angulaire}
Lorsqu'un mobile est en MCU, il peut être plus intéressant de donner sa vitesse en fonction de l'angle parcouru à chaque seconde. On parle alors de \motcle{vitesse angulaire}. Cet angle est généralement donné en radian.
\begin{encadre}
      \motcle{Vitesse angulaire} : \(\omega [rad \cdot s^{-1}]\)
\end{encadre}

\begin{itemize}[label=\(\rightarrow\)]
      \item Si un mobile possède une vitesse angulaire de \(\omega [rad \cdot s^{-1}]\), quel est l'angle balayé après une durée égale à \(T[s]\) ?
            \pointilles{2}
      \item Isole la période dans cette expression
            \pointilles{2}
      \item Isole la période dans l'équation de la vitesse tangentielle.
            \pointilles{2}
      \item Trouve l'expression permettant de calculer la vitesse angulaire à partir de la vitesse tangentielle.
            \pointilles{2}
\end{itemize}

\newpage

\section{Exercices}
\begin{exercise}
      Un mobile décrit un MCU avec un rayon de 5[m] en 20[s]. Quelle est sa vitesse angulaire ?
\end{exercise}
\begin{solution}
      \(\omega=0,3142[rad/s]\)
\end{solution}

\begin{exercise}
      Quelle est la fréquence d'un mobile dont la vitesse angulaire vaut 7[rad/s] ?
\end{exercise}
\begin{solution}
      \(f=1,114[Hz]\)
\end{solution}

\begin{exercise}
      La période de rotation d'un mobile est de 15[s]. Que vaut sa vitesse angulaire ? Si le rayon est de 20[m], que vaut la vitesse tangentielle ?
\end{exercise}
\begin{solution}
      \begin{itemize}
            \item \(\omega=0,4189[rad/s]\)
            \item  \(v=8,3778[m/s]\)
      \end{itemize}
\end{solution}

\begin{exercise}
      La vitesse tangentielle d'un mobile est de 36[km/h], si le rayon est de 75[m], que valent la fréquence et la vitesse angulaire ?
\end{exercise}
\begin{solution}
      \begin{itemize}
            \item \(\omega=0,13333[rad/s]\)
            \item  \(f=0,02122[m/s]\)
      \end{itemize}
\end{solution}