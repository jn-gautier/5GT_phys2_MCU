\chapter{Force centripète}
Nous avons vu dans le chapitre sur les lois de newton qu'une accélération est le produit d'une force s'exerçant sur une masse. L'accélération centripète du MCU n'échappe pas à cette loi.
Il n'est pas toujouurs nécessaire de connaitre la nature de cette force, il peut s'agir d'une force magnétique, de l'attraction d'une planète, d'une traction, de frottements, d'une partie du poids, ....

Dans la relation \(a=\frac{F}{m}\), la masse n'est pas un vecteur. Cela implique que les vecteurs \(\vec{a}\) et \(\vec{F}\) ont toujours le même sens et la même direction. On en déduit naturellement que la force qui produit un MCU est toujours une \motcle{force centripète}.
la valeur de la force centripète est naturellement donnée par :
\begin{mytheo*}{Force centripète}
    \begin{equation}
        F_c=\frac{m \cdot v^2}{r}
    \end{equation}
\end{mytheo*}
\newpage

\section{Virage relevé}
Dans les applications du MCU qui ont été abordées, une voiture dans un virage est un exemple fréquemment rencontré. Jusqu'à présent, nous avons considéré que les forces centripètes permettant à une voiture de tourner étaient les frottements entre les roues et le sol. Toutefois, ceux-ci ne sont pas toujours suffisants : les voitures peuvent circuler à une vitesse importante, à la sortie d'une autoroute, par exemple. Les frottements peuvent aussi être fortement diminués à cause de la pluie ou de la neige.

Il est toutefois possible de créer une force centripète qui provient du poids du véhicule, cela nécessite que le virage soit incliné, on parle alors de \motcle{virage relevé}.

\begin{figure}[h]
    \begin{minipage}{.5\textwidth}
        \centering
        \includegraphics[width=.8 \linewidth]{virage_relevé.jpg}
        \caption{Un exemple de virage relevé.}
    \end{minipage}
    \begin{minipage}{.5\textwidth}
        \centering
        \includegraphics[width=.8 \linewidth]{virage_relevé_II.jpg}
        \caption{Un autre exemple de virage relevé.}
    \end{minipage}
\end{figure}

\begin{tcolorbox}[title={Réalise, ci-dessous, un schéma \enquote{vue de haut} et \enquote{vue en coupe} des forces agissant sur une voiture dans un virage relevé.},sidebyside,height=6cm]

    \tcblower

\end{tcolorbox}

\newpage

Dans cette situation, les deux seules forces qui s'exercent verticalement sont le poids du véhicule et la composante en Y de la force normale. Donc :
\begin{equation}
    m \cdot g = F_N \cdot cos(\theta)
    \label{eqn:bilan_vertical}
\end{equation}

On sait aussi que la force centripète correspond à la composante horizontale de la force normale. Donc :
\begin{equation}
    \frac{m \cdot v^2}{r} = F_N \cdot sin(\theta)
    \label{eqn:bilan_horizontal}
\end{equation}

Si on associe les équations \ref{eqn:bilan_vertical} et \ref{eqn:bilan_horizontal}, on peut écrire que :
\begin{equation}
    \frac{m \cdot v^2}{r \cdot sin(\theta)} = \frac{m \cdot g}{cos(\theta)}
\end{equation}
On peut réarranger cette équation pour obtenir la vitesse maximale avec laquelle le virage peut être emprunté :
\begin{mytheo*}{Vitesse maximale dans un virage relevé}
    \begin{equation}
        v=\sqrt{r \cdot g \cdot tan \theta}
    \end{equation}
\end{mytheo*}
On remarquera que cette vitesse ne dépend pas de la masse du véhicule.

\newpage

\section{Exercices}
%%%%%%%%%%%%%%%%%%%%
\begin{exercise}%1
    Quelle est la vitesse maximale avec laquelle une voiture de 300[kg] peut aborder un virage relevé de \(10^{\circ}\) si le rayon de courbure de celui-ci est de 25[m]?
\end{exercise}
\begin{solution}%1
    \(v=6,5760[m/s]\)
\end{solution}

%%%%%%%%%%%%%%%%%%%%%
\begin{exercise}
    À la sortie d'une autoroute, on souhaite que les voitures puissent prendre un virage en roulant à du 80[km/h]. Si la longueur totale du virage est de 100[m] et qu'il permet de faire un 1/4 de tour, quel est son angle de relèvement ?
\end{exercise}
\begin{solution}
    \(\theta=38,334 ^{\circ}\)
\end{solution}

%%%%%%%%%%%%%%%%%%%%%%
\begin{exercise}
    Une masse \(m=0,2[kg]\) est attachée au bout d'une corde de longueur \(L=0,7[m]\). On fait tourner la masse à une vitesse de \(v=6m/s\).
    \begin{enumerate}[a]
        \item Quel est l'angle formé par la corde par rapport à sa position de repos verticale.
        \item Quelle est la tension dans la corde à ce moment, la somme de \(T_X\) et \(T_Y\)?
    \end{enumerate}
\end{exercise}
\begin{solution}

\end{solution}
